\chapter{Background}
As mentioned in Chapter~\ref{chapter:intro}, JavaScript has several shortcomings. This chapter will explain those shortcomings in more detail. It will also introduce two of the tools that will be used in this project; Haste~\cite{haste-lang} and JSFlow~\cite{jsflow,jsflow-csf12,jsflow-sac14}. Finally, some terminology and related work will be presented.

\section{Problems with JavaScript - part deux}
As mentioned in Chapter~\ref{chapter:intro}, from a security standpoint JavaScript suffers from weak dynamic typing and having access to sensitive information in the browser.
\subsection{Weak dynamic typing}
\label{chapter:weak-dynamic-typing}
There are several odd features one can use in JavaScript due to the weak dynamic typing. Where a statically typed language (e.g. Java) would give a compile error, the JavaScript code will be run and perhaps succeed but with unexpected results. Figure~\ref{fig:error_java} shows a logical error that is caught at compile time in Java. If the code in Figure~\ref{fig:error_java} would be allowed to run, it would be up to the runtime environment to determine how the code would be interpreted. The same error in JavaScript is shown in Figure~\ref{fig:error_js}. Since JavaScript does not have any static type checking the decision on how to interpret the code will be made at runtime. Instead of failing and raising a runtime error, JavaScript will convert the \textbf{true} value into a number which in this case corresponds to the number \textbf{1}. Hence the addition will be evaluated to \textbf{3}. Situations like that, where a clear type error is allowed to propagate and potentially change the state of the application should be considered dangerous and prone to producing bugs.

Another issue with the weak type system in JavaScript can be seen in Figure~\ref{fig:js_comparison}. It is perfectly legal to compare a function with an array in JavaScript. The code in Figure~\ref{fig:js_comparison} will evaluate to \textbf{true}. This can in turn make it very difficult to find potential bugs since JavaScript will try to convert values of different types to the same types and then make the comparison. In e.g. Java, the comparison in Figure~\ref{fig:js_comparison} would not go through the type checker.

In principle, weak typing is when a programmer can mix different types. In some cases it can make sense to allow it, e.g. adding an integer and a float value. It is not only in JavaScript that the examples in Figure~\ref{fig:error_js} and Figure~\ref{fig:js_comparison} would go through a "compilation phase". Other dynamic languages such as Erlang, Python and Ruby would allow those examples through the compiler. However, Erlang, Python and Ruby all have type checking at runtime and those examples would generate an error at runtime.

One could argue that it is rather silly examples. Who would ever compare a function with an array? The example in Figure~\ref{fig:js_bad_age} shows exactly why weak typing is a bad thing. Assume that a user inputs \emph{Alice} as the name and \emph{42} as the age. The example in Figure~\ref{fig:js_bad_age} would gladly write
\begin{center}
  \emph{"Alice is now 42 years old. In 20 years Alice will be 4220 years old!"}
\end{center}
to the console. This is because {\tt +} is not only addition, it is also concatenation and if one of the operands is a string, {\tt +} will always convert the second operand to a string and do a concatenation.

\begin{figure}[h]
  \begin{verbatim}
    int a = 2;
    boolean b = true;
    a + b;  // This should not be allowed to be run
  \end{verbatim}
  \caption{Logical error in Java}
  \label{fig:error_java}
\end{figure}
\begin{figure}[h]
  \begin{verbatim}
    var a = 2;
    var b = true;
    a + b;  // This will evaluate to 3
  \end{verbatim}
  \caption{Logical error in JavaScript}
  \label{fig:error_js}
\end{figure}
\begin{figure}[h]
  \begin{verbatim}
    (function(x) { return x * x; }) > [1,2,3];
  \end{verbatim}
  \caption{Weird comparison in JavaScript}
  \label{fig:js_comparison}
\end{figure}
\begin{figure}[h]
  \begin{verbatim}
    var name = prompt("What is your name?", "");
    var age = prompt("What is your age?", "");
    console.log(
        name + " is now " + age + " years old. In 20 years " +
        name + " will be " + (age + 20) + " years old!"
    );
  \end{verbatim}
  \caption{Weak types in JavaScript}
  \label{fig:js_bad_age}
\end{figure}

\subsubsection{Attempts to solve the type problem}
There have been several attempts of providing a more secure type system to JavaScript, everything from creating a statically typed language that compiles to JavaScript to a compiler from an already existing programming language and compile it to JavaScript to a static type checker. Examples of attempted soultions are
\begin{itemize}
  \item \textbf{TypeScript}~\cite{typescript}, a typed superset of JavaScript that compiles to plain JavaScript.
  \item \textbf{TeJaS}~\cite{tejas-art,tejas-git}, which allows you to annotate type signatures as comments in the JavaScript code and then type checks the code.
  \item \textbf{GHCJS}~\cite{ghcjs} and \textbf{Haste}~\cite{haste-lang,haste-symposium}, which compiles from Haskell, a statically typed, high-level functional programming language~\cite{haskell}, to JavaScript.
\end{itemize}

\subsection{Sensitive information in the browser}
The browser has access to several different types of sensitive information and can be used by an attacker to get the sensitive information. JavaScript can gain access to e.g. cookies, send HTTP requests and make arbitrary DOM (\emph{Document Object Model}) modifications. If untrusted JavaScript is executed in a victim's browser, the attacker can, among other things, perform the following attacks:
\begin{itemize}
  \item \textbf{Cookie theft}, where the attacker can gain access to the victim's cookies that are associated with the current website.
  \item \textbf{Keylogging}, where the attacker can create and register a keyboard event listener and send the keystrokes to the attacker's own server in order to potentially record passwords, credit card information etc.
  \item \textbf{Phishing}, where the attacker can insert fake forms by manipulating the DOM and fool the user to submit sensitive information which will be redirected to the attacker.
\end{itemize}
\subsubsection{Attempts to solve the information problem}
The current solution to secure the sensitive information JavaScript has access to as of now is to sandbox the script and run it in a secure environment. There are mainly three ways of doing this.
\begin{itemize}
  \item Wrap the script inside a call to {\tt with} and pass a faked {\tt window} object and execute the code with {\tt eval}.
  \item Use an iframe and set the sandbox attribute to either not allow scripts to run or allow the scripts to run within that iframe only. Unfortunately, as explained in~\cite{js_in_js}, iframes have some issues as well.
  \item Use existing tools to help sandbox third party code, such as:
    \begin{itemize}
      \item \textbf{JavaScript in JavaScript}~\cite{js_in_js}, an interpreter that allows an application to execute third-party scripts in a completely isolated, sandboxed environment.
      \item \textbf{Caja}~\cite{caja_spec}, a compiler to make third party code safe to embed within a website.
    \end{itemize}
\end{itemize}
\section{Haste}
Haste (\emph{HASkell To Ecmascript compiler}) is a compiler that compiles the high-level language Haskell to JavaScript. The Haste compiler is plugged into the compilation process from GHC, the Glasgow Haskell Compiler. As can be seen in Figure~\ref{fig:system}, Haste starts its compilation process after GHC has done some code optimization. From the optimized code from GHC, Haste will create an AST, \emph{Abstract Syntax Tree}, for JavaScript. The AST will then be optimized and after the optimization process in Haste is done the actual JavaScript code will be generated.
\begin{figure}[h]
  \begin{tabular}{|c|c|c|}
    \hline
    Step & Operation & GHC/Haste \\
    \hline
    1 & Parse & GHC \\
    2 & Type check & GHC \\
    3 & Desugar & GHC \\
    4 & Intermediate code generation & GHC \\
    5 & Optimization & GHC \\
    6 & Intermediate code generation (JS AST) & Haste \\
    7 & Optimization & Haste \\
    8 & Code generation to JavaScript & Haste \\
    \hline
  \end{tabular}
  \caption{The compilation process for the Haste compiler}
  \label{fig:system}
\end{figure}

When compiling the Haskell source code with Haste, the compilation process will result in a JavaScript file. If Haste.App, a client-server communication framework for Haste, is used, the compilation process will create the source code for the client in JavaScript and a server binary~\cite{haste-symposium}.

\section{JSFlow}
JSFlow is an interpreter written in JavaScript that dynamically checks the JavaScript code at runtime to ensure information flow security. Currently, JSFlow supports full information flow control for ECMA-262 v5, the standard which JavaScript is built upon, apart from strict mode and JSON (\emph{JavaScript Object Notation}).

Within information flow security, there are two types of flow that must be checked - explicit flow and implicit flow. Even though there are several different ways an attacker can gain information about a system (e.g. via timing attacks where the attacker analyzes the computation time to gain information about the system), only explicit and implicit flows for computations are considered. JSFlow does not provide security for e.g. timing attacks and due to this, handling timing attacks and other side-channel attacks will be outside of the scope for this thesis.
\subsection{Explicit flow}
With explicit flow, one means when a data in a \emph{high} context leaks information to a \emph{low} context explicitly. An example can be seen in Figure~\ref{fig:expflow} where the value of the high variable {\tt h} is leaked to the low variable {\tt l}. Obviously this should be illegal when information flow security is applied and explicit flows are not difficult to find when dynamically checking the code. When data is written to a variable, one simply must keep track of the context of the variable and the context of the data. If the variable is in low context and the data is in high context an error should be produced and execution of the JavaScript code should be stopped. All other scenarios (high variable with high data, high variable with low data and low variable with low data) are allowed.
\subsection{Implicit flow}
\label{chapter:implicit_flow}
Implicit flows occurs when e.g. a language's control structure is used in combination with side effects to leak information. Figure~\ref{fig:impflow} shows an example of an implicit flow. The variable {\tt l} will get the value 1 if and only if the variable {\tt h} is an odd number. Otherwise {\tt l} will have the value 0. A dynamic system that will handle implicit flows must associate a security context with the control flow~\cite{jsflow-csf12}. In Figure~\ref{fig:impflow}, the body of the if statement should be executed in a secure context and therefore the variable {\tt l} must be a secure variable in order for the flow to be valid.

\begin{figure}[h]
  \captionsetup[subfigure]{singlelinecheck=off,justification=raggedright}
  \begin{subfigure}[b]{0.5\textwidth}
    \begin{verbatim}
      l := h
    \end{verbatim}
    \caption{Explicit flow}
    \label{fig:expflow}
  \end{subfigure}
  \begin{subfigure}[b]{0.5\textwidth}
    \begin{verbatim}
      h := h mod 2;
      l := 0;
      if h = 1 then l := 1;
      else skip;
    \end{verbatim}
    \caption{Implicit flow}
    \label{fig:impflow}
  \end{subfigure}
  \caption{Implicit and explicit flow}
\end{figure}

\subsection{Example of coding for JSFlow}
When creating a web application in JavaScript that JSFlow should be able to check, there is only one function that the programmer must know about - the {\tt upg} function. The function {\tt upg} is used when lifting a computation into a high context. In Figure~\ref{fig:upg}, the variable {\tt h} is lifted into a high context by calling {\tt upg} on the data to be stored to {\tt h}. Due to the call to {\tt upg}, the variable {\tt h} will be a high variable containing the number 42. As can be seen with the variables {\tt l}, which is a low variable, and {\tt t}, which is a high variable, the default level for a compuation is low unless JSFlow infers that a variable \textbf{must} be high due to part of the compuation being high.

\begin{figure}[h]
  \begin{verbatim}
    // Variable l is a low variable of value 2
    var l = 2;

    // Variable h is a high variable of value 42
    var h = upg(42);

    /* Variable t must be a high variable due to h
       being high */
    var t = l + h;
  \end{verbatim}
  \caption{Creating high and low variables in JavaScript with JSFlow}
  \label{fig:upg}
\end{figure}
The {\tt upg} function will take a computation (in Figure~\ref{fig:upg} the compuation is simply the number 42) and bind that value to the assigned variable (in Figure~\ref{fig:upg} the variable {\tt h}) and put the variable in a high context.

\subsection{Flows in a pure functional language}
Even though there are two different flows in JavaScript (implicit and explicit flows), there is only one type of flow in a pure functional programming language like Haskell, an explicit flow~\cite{seclib}. As described in Chapter~\ref{chapter:implicit_flow}, an implicit flow depends on control structures in combination with side effects in order to leak information. However, even though a pure functional programming language like Haskell contains control structures, a pure function does not contain side effects. A control structure like an if-statement can be interpreted as a regular function returning a constant value. This means that a function like
\begin{verbatim}
    f :: HInt -> LInt
    f x = if x `mod` 2 == 0
              then 42
            else -42
\end{verbatim}
will look like an implicit flow but is in fact an explicit flow. Note that, in this case {\tt HInt} stands for a \emph{High Int} and {\tt LInt} stands for a \emph{Low Int}. An if-then-else can be rewritten as a regular function~\cite{if-then-else}
\begin{verbatim}
    myIf :: Bool -> a -> a -> a
    myIf True  b1 _ = b1
    myIf False _ b2 = b2
\end{verbatim}
where {\tt b1} and {\tt b2} are the different branches. The different branches are of type {\tt a}, which in this case can be any arbitrary expression. Rewriting the function {\tt f} using {\tt myIf} can be done as follows:
\begin{verbatim}
    f :: HInt -> LInt
    f x = myIf (x `mod` 2 == 0) 42 (-42)
\end{verbatim}
where a constant value, either {\tt 42} or {\tt -42}, is returned. Since there are no side effects in a pure function like {\tt f}, there can be no implicit flows. Every flow will be explicit, which in turn makes it easier to create a structure in the type system that keeps track of the flow.

\section{Non-interference}
The principles of non-interference were introduced by Goguen and Meseguer in 1982~\cite{non-interference-goguen-meseguer} who defined non-interference to be
\newline
\begin{center}
  \emph{"one group of users, using a certain\\set of commands, is noninterfering\\with another group of users if what\\the first group does with those\\commands has no effect on what the\\second group of users can see."}
\end{center}
When talking about non-interference with regards to information flow control, one means a property that states that the public outcome does not depend on any private input. An attacker should not be able to distinguish between two computations from their outputs if the computations only vary in the secret input.

As an example of non-interference, assume the following function where something of type \textbf{Char} is of high value and something of type \textbf{Int} is of low value:
\begin{verbatim}
    fOk :: (Char, Int) -> (Char, Int)
    fOk (c, i) = (chr (ord c + i), i + 42)
\end{verbatim}
The function {\tt fOk} preserves confidentiality since it does not leak any valuable information about the value {\tt c}. In this case, preserving the confidentiality of {\tt c} means that no information of {\tt c} is leaked. It is said to be \emph{non-interfered} since the public result (the Int value) is independent of the value of {\tt c}. If the function instead was defined as
\begin{verbatim}
    fBad1 :: (Char, Int) -> (Char, Int)
    fBad1 (c, i) = (c, ord c)
\end{verbatim}
it would not be non-interfered because the confidentiality is broken. Information about {\tt c} is leaked through the low Int and the corresponding decimal value of the ASCII number of {\tt c} is returned as the second value of the tuple.

Unfortunately, information leakage is seldom as explicit as in the {\tt fBad1} function. An attacker might be clever and attempt
\begin{verbatim}
    fBad2 :: (Char, Int) -> (Char, Int)
    fBad2 (c, i) = (c, if ord c > 31 then 1 else 0)
\end{verbatim}
which would give information about whether or not the variable {\tt c} is a printable character (the ASCII values of printable characters start at 32~\cite{ascii})~\cite{seclib}. Just as with {\tt fBad1}, {\tt fBad2} does not satisfy the non-interference property.

\section{Declassification}
\label{chapter:declassification}
A system that satisfies the non-interference policy is a very strict system. A system usually needs some kind of controlled release of confidential data. As an example, imagine a login system. A user's password should be handled as confidential data when the user attempts to login. If the login succeeded the user should be authenticated and redirected whereas if the login failed the user should be prompted with a message saying the username/password combination was incorrect. If the system was non-interfered, there could be no message explaining to the user that the username/password was incorrect since that output would rely on confidential data. Another example could be when a credit card is being used for online payment. In the order receipt it is not uncommon to include the last four digits of the credit card number. Again, this is something that can not be done if the system is non-interfered since a credit card number should be considered confidential.

Unfortunately, there is no way for the non-interference policy to distinguish between intended release of information and release that occurs due to an attack or programming error. In order to allow controlled information leak one can use declassification policies.
Taking the example of the credit card explained above and assuming a function {\tt getLastFour} that takes a secret credit card number {\tt h} as an argument and returns a secret value containing the last four digits of the credit card number, producing an intended information release to the variable {\tt l} can be achieved by calling a {\tt declassify} function:
\begin{verbatim}
    l := declassify(getLastFour(h))
\end{verbatim}
However, just allowing declassification everywhere can be dangerous. In theory, an attacker could compromise the declassification and extract more information than intended. Due to this, work on classifying the declassification into four different dimensions has been presented in~\cite{declassification-dimensions}. The proposed dimensions are
\begin{itemize}
  \item \textbf{What} information is released.
  \item \textbf{Who} controls the information release.
  \item \textbf{Where} in the system the information is released from.
  \item \textbf{When} the information is released.
\end{itemize}
and these should be seen as recommendations for how to build the declassification policies in systems.
\section{Related Work}
\label{chapter:related}
There has been research within information flow control and libraries have been created in order to help enforce information flow policies and secure both confidentiality and integrity of the information. Some of the most relevant findings for this project will be described below.
\subsection{Labeled IO}
Labeled IO (\emph{LIO}) is a library created in Haskell for dynamic information flow control~\cite{lio-2011}. Compared to the library created for this thesis, LIO keeps track of a \emph{current label}, which is an upper bound of the labels of all data that can be observed or modified in the evaluation context. One can also bound the current label with a \emph{current clearance}, where the clearance of a region of code can be set in advance to provide an upper bound of the current label within that region. LIO attempts to close the gap with static analysis, which even though it has its advantages (fewer run-time failures, reduced run-time overhead etc.) has a problem when new kinds of data (e.g. user input) can be encountered at runtime, and dynamic systems.
\subsection{Seclib}
Seclib is a light-weight library for information flow control in Haskell~\cite{seclib, seclib_git}. Just as with the library created in this thesis, Seclib is based on monads and all private data lives within the created monad. However, compared to the library in this thesis, Seclib creates two different monads, \textbf{Sec} and \textbf{SecIO}, where SecIO is an extended IO monad (an IO monad wrapped in a Sec monad). Seclib was designed to be a small, lightweight library and consists, as of Jan. 29 2015, of only 342 lines of code.
\subsection{LMonad}
LMonad~\cite{lmonad} is a library tailored to provide information flow control for web applications written in Yesod, a framework for creating web applications in Haskell~\cite{yesod}. It is a generalization of LIO and provides a Monad Transformer~\cite{monad-transformer}, which in theory means it can be "wrapped around" any monad. Programmers can define their own information flow control policies and after defining them, LMonad guarantees that database interactions follow the policies.
\subsection{RDR}
RDR is a monad for Restricted Delegation and Revocation which builds on the DLM (\emph{Decentralized Label Model}) to provide information flow control. DLM allows declassification in a decentralized way~\cite{dlm} and consists of four important parts~\cite{dlm-site}:
\begin{itemize}
  \item \emph{Principles}, which is an entity with power to change and observe certain aspects of the system. A principle \emph{p} can delegate authority to a principle \emph{q} and the actions taken by \emph{q} is implicitly assumed to be authorized by \emph{p}. The principals can have different security policies which makes the system modular.
  \item \emph{Reader policy}, which allows the owner of the policy to specify which principals who are allowed to read a given piece of information. This is a \emph{confidentiality policy}.
  \item \emph{Writer policy}, which allows the owner of the policy to specify which principals who may have influenced the value of a given piece of information. This is an \emph{integrity policy}.
  \item \emph{Labels}, which is a pair of a confidentiality policy and an integrity policy. It is the labels that are used to anotate programs.
\end{itemize}

The purpose of RDR is to extend restricted delegation and revocation to information flow control. The main principle is to allow information flow to a predefined chain of principals, but keeping a right to revoke it at any time. The RDR monad is built by using a Reader monad over the IO monad. One of the ideas of RDR is to see restricted delegation as declassification and protected values should only be allowed to be read by using a valid principal and due to the DLM, mutually-distrusting principals should be allowed to express individual confidentiality and integrity policies~\cite{rdr}.
\subsection{Why another library?}
As explained above, there are already working libraries for information flow control so why would another library be needed? Even though the current libraries are good, they lack one part - integration with a dynamic system. The library will help close the gap between static control for information flow control and already created tools for dynamic check of information flow control (JSFlow) and created tools for securing JavaScript from the viewpoint of a type system (Haste).

% The created library will be able to be run in GHC if one wants to create a system strictly for Haskell but it should also be able to run the code with Haste and as a result get produced JavaScript code which is tagged so JSFlow can dynamically track the information flow.  <-- För abstract..? Eller kanske introduction?
