\chapter{Testing of the implementation}
The testing part of the project was done in two different ways. First, randomized tests of SwapIFC were done using QuickCheck~\cite{quickcheck, quickcheck-wiki}. Randomized testing is an area where a lot of research has been made.
\section{Testing of the library with QuickCheck}
In order to test SwapIFC with QuickCheck, every Flow instance and every operator needed to be modeled as a constructor of a datatype. This is to help randomize both the instance class and the operator. The probability distribution for each class is the same and given a class, the probability distribution for each operation is the same. The general algorithm for the tests can be seen in Figure~\ref{fig:testing_swapifc}. Note that the \emph{oneof} function and the \emph{generateXFlows} functions are random functions and the \emph{validate} will return true if and only if the result of the operand \emph{op} applied to the flows in \emph{flows} will generate the same value as the normal operand corresponding to \emph{op} applied to the normal values corresponding to \emph{flows} and that the tag is being preserved. The preservation of the tag means that if any of the operands have a High tag, then the result must be High. An example of this is the following:
\begin{center}
  \begin{lstlisting}[language=Haskell]
    let a = mkHigh 42
    let b = mkLow 10
    let c = a .+. b
    return (tag of c == High && value of c == 42 + 10)

    let d = mkLow 42
    let e = mkLow 10
    let f = a .<. b
    return (tag of f == Low && value of f == 42 < 10)
  \end{lstlisting}
\end{center}

The tags are only part of the type system and because of this some of the unsafe functions must be used in order to validate the tags. The unsafe functions used in the tests are \emph{unwrapValue} in order to check the computed flow value and \emph{unsafeShow} in order to get a string representation of the flow value. The string representation will be the value followed by "\_$<$T$>$", where T is either \emph{H} or \emph{L} for High and Low respectively. So in the example above, \emph{unsafeShow a} will yield the string "42\_$<$H$>$" and \emph{unsafeShow d} will yield the string "42\_$<$L$>$".

The algorithm is repeated 10000 times and the source code can be found in Appendix~\ref{app:tests}.

\begin{figure}[h]
  \begin{algorithm}[H]
    \caption{Testing SwapIFC algorithm}
    \begin{algorithmic}
      \State $c\gets oneof([Num, Frac, Bool, Eq, Ord])$
      \If {$c == Num$}
          \State $op\gets oneof([Add, Sub, Mul, Neg, Abs, Sig])$
          \State $flows\gets generateNumFlows$
      \EndIf        
      \If {$c == Frac$}
          \State $op\gets oneof([Div, Recip, FromRational])$
          \State $flows\gets generateFracFlows$
      \EndIf
      \If {$c == Bool$}
          \State $op\gets oneof([And, Or, Not])$
          \State $flows\gets generateBoolFlows$
      \EndIf
      \If {$c == Eq$}
          \State $op\gets oneof([Equals, NotEquals])$
          \State $flows\gets generateEqFlows$
      \EndIf
      \If {$c == Ord$}
          \State $op\gets oneof([Less, GreaterEquals, LessEquals, Greater, Max, Min])$
          \State $flows\gets generateOrdFlows$
      \EndIf
      \State
      \State
      \Return Validate flows op
    \end{algorithmic}
  \end{algorithm}
  \caption{Pseudo code for the testing of SwapIFC}
  \label{fig:testing_swapifc}
\end{figure}

\section{Testing of the code generation}
aosifjasofijasfoij
\begin{figure}[h]
  \begin{algorithm}[H]
    \caption{Testing JSFlow compatibility algorithm}
    \begin{algorithmic}
      \If {$c == 42$}
         \State $a\gets 10$
      \EndIf
    \end{algorithmic}
  \end{algorithm}
  \caption{Pseudo code for the testing of JSFlow}
  \label{fig:testing_jsflow}
\end{figure}
asodijasodijsd

\section{Argument for correctness of the library}
